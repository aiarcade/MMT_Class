\documentclass[11pt,a4paper]{article}

% Packages
\usepackage[utf8]{inputenc}
\usepackage[margin=1in]{geometry}
\usepackage{graphicx}
\usepackage{booktabs}
\usepackage{array}
\usepackage{tabularx}
\usepackage{longtable}
\usepackage{multirow}
\usepackage{enumitem}
\usepackage{xcolor}
\usepackage{titlesec}
\usepackage{fancyhdr}
\usepackage{hyperref}
\usepackage{colortbl}

% Colors
\definecolor{headerblue}{RGB}{0,51,102}
\definecolor{lightblue}{RGB}{230,240,250}
\definecolor{accentgreen}{RGB}{0,102,51}

% Header/Footer
\pagestyle{fancy}
\fancyhf{}
\fancyhead[L]{\small\textcolor{headerblue}{CXT332 - Multimedia Technologies}}
\fancyhead[R]{\small\textcolor{headerblue}{Reorganized Syllabus}}
\fancyfoot[C]{\thepage}
\renewcommand{\headrulewidth}{0.5pt}

% Title formatting
\titleformat{\section}{\Large\bfseries\color{headerblue}}{\thesection}{1em}{}[\titlerule]
\titleformat{\subsection}{\large\bfseries\color{accentgreen}}{\thesubsection}{1em}{}

% Custom commands
\newcommand{\modulebox}[2]{%
    \noindent\colorbox{headerblue}{%
        \parbox{\dimexpr\textwidth-2\fboxsep}{%
            \textcolor{white}{\textbf{#1}} \hfill \textcolor{white}{\textbf{#2 Hours}}%
        }%
    }%
}

\begin{document}

% Title Page
\begin{titlepage}
    \centering
    \vspace*{2cm}
    
    {\Huge\bfseries\textcolor{headerblue}{MULTIMEDIA TECHNOLOGIES}\\[0.5cm]}
    {\Large\textcolor{accentgreen}{CXT 332}\\[1cm]}
    
    \rule{\textwidth}{1pt}\\[0.5cm]
    
    {\LARGE\bfseries Reorganized Syllabus\\[0.3cm]}
    {\large\textit{Optimized 4-Module Structure}\\[0.5cm]}
    
    \rule{\textwidth}{1pt}\\[2cm]
    
    \begin{tabular}{|c|c|c|c|c|}
        \hline
        \rowcolor{lightblue}
        \textbf{Category} & \textbf{L} & \textbf{T} & \textbf{P} & \textbf{Credits} \\
        \hline
        PEC & 2 & 1 & 0 & 3 \\
        \hline
    \end{tabular}
    
    \vspace{1cm}
    
    \begin{tabular}{|l|l|}
        \hline
        \rowcolor{lightblue}
        \textbf{Total Contact Hours} & 36 Hours \\
        \hline
        \textbf{Number of Modules} & 4 Modules (9 Hours Each) \\
        \hline
        \textbf{Prerequisite} & NIL \\
        \hline
    \end{tabular}
    
    \vfill
    
    {\large Department of Computer Science and Design\\[0.3cm]}
    {\large APJ Abdul Kalam Technological University\\[0.5cm]}
    
    \vspace{1cm}
\end{titlepage}

% Table of Contents
\tableofcontents
\newpage

%----------------------------------------------------------------------
\section{Preamble}
%----------------------------------------------------------------------

This course provides a comprehensive understanding of multimedia systems, covering the fundamental concepts of digital media representation, compression techniques, and modern applications. The reorganized structure ensures a logical progression from basics to advanced topics, with practical emphasis on industry-relevant standards.

\subsection{Key Learning Objectives}
\begin{itemize}[leftmargin=*]
    \item Understand multimedia data representations including graphics, audio, and video
    \item Master both lossless and lossy compression algorithms
    \item Apply knowledge of industry standards (JPEG, MPEG family)
    \item Explore modern applications including content-based retrieval and cloud computing
\end{itemize}

%----------------------------------------------------------------------
\section{Course Outcomes}
%----------------------------------------------------------------------

\begin{longtable}{|c|p{12cm}|}
    \hline
    \rowcolor{headerblue}
    \textcolor{white}{\textbf{CO\#}} & \textcolor{white}{\textbf{Course Outcome Description}} \\
    \hline
    \endfirsthead
    \hline
    \rowcolor{headerblue}
    \textcolor{white}{\textbf{CO\#}} & \textcolor{white}{\textbf{Course Outcome Description}} \\
    \hline
    \endhead
    
    CO1 & Describe the basic concepts of multimedia data representations, color models, audio and video signals. \textit{(Cognitive Level: Understand)} \\
    \hline
    CO2 & Apply lossless and lossy compression algorithms for developing multimedia applications. \textit{(Cognitive Level: Apply)} \\
    \hline
    CO3 & Analyze image, audio, and video compression standards and their applications. \textit{(Cognitive Level: Analyze)} \\
    \hline
    CO4 & Evaluate content-based retrieval techniques and cloud computing applications in multimedia. \textit{(Cognitive Level: Evaluate)} \\
    \hline
\end{longtable}

%----------------------------------------------------------------------
\section{CO-PO Mapping}
%----------------------------------------------------------------------

\begin{center}
\small
\begin{tabular}{|c|c|c|c|c|c|c|c|c|c|c|c|c|}
    \hline
    \rowcolor{lightblue}
    & \textbf{PO1} & \textbf{PO2} & \textbf{PO3} & \textbf{PO4} & \textbf{PO5} & \textbf{PO6} & \textbf{PO7} & \textbf{PO8} & \textbf{PO9} & \textbf{PO10} & \textbf{PO11} & \textbf{PO12} \\
    \hline
    \textbf{CO1} & 3 & 2 & 1 & - & 2 & - & - & - & - & - & - & 2 \\
    \hline
    \textbf{CO2} & 3 & 3 & 2 & 2 & 2 & - & - & - & 1 & - & - & 2 \\
    \hline
    \textbf{CO3} & 3 & 2 & 2 & 2 & 3 & - & - & - & 1 & - & - & 2 \\
    \hline
    \textbf{CO4} & 2 & 2 & 2 & 2 & 3 & 1 & 1 & - & 1 & 1 & 1 & 3 \\
    \hline
\end{tabular}
\end{center}

\vspace{0.5cm}
\noindent\textbf{PO Definitions:}
\begin{center}
\footnotesize
\begin{tabular}{|l|l||l|l|}
    \hline
    \rowcolor{lightblue}
    \textbf{PO\#} & \textbf{Description} & \textbf{PO\#} & \textbf{Description} \\
    \hline
    PO1 & Engineering Knowledge & PO7 & Environment and Sustainability \\
    \hline
    PO2 & Problem Analysis & PO8 & Ethics \\
    \hline
    PO3 & Design/Development of Solutions & PO9 & Individual and Team Work \\
    \hline
    PO4 & Conduct Investigations & PO10 & Communication \\
    \hline
    PO5 & Modern Tool Usage & PO11 & Project Management and Finance \\
    \hline
    PO6 & The Engineer and Society & PO12 & Lifelong Learning \\
    \hline
\end{tabular}
\end{center}

\newpage

%======================================================================
\section{Detailed Syllabus (4 Modules - 36 Hours)}
%======================================================================

%----------------------------------------------------------------------
\subsection*{}
\modulebox{MODULE 1: Multimedia Fundamentals \& Data Representation}{9}
\addcontentsline{toc}{subsection}{Module 1: Multimedia Fundamentals \& Data Representation (9 Hours)}

\vspace{0.5cm}

\noindent\textbf{Overview:} This module introduces the foundational concepts of multimedia systems, including data types, file formats, color science, and digital audio/video basics.

\vspace{0.3cm}

\begin{longtable}{|c|p{10cm}|c|}
    \hline
    \rowcolor{lightblue}
    \textbf{Unit} & \textbf{Topic} & \textbf{Hours} \\
    \hline
    \endfirsthead
    \hline
    \rowcolor{lightblue}
    \textbf{Unit} & \textbf{Topic} & \textbf{Hours} \\
    \hline
    \endhead
    
    1.1 & \textbf{Introduction to Multimedia Systems} \newline Multimedia concepts, Hypermedia, WWW and Internet infrastructure & 1 \\
    \hline
    1.2 & \textbf{Multimedia Software Tools} \newline Editing tools, Authoring tools, Production workflows & 1 \\
    \hline
    1.3 & \textbf{Graphics \& Image Data Types} \newline Raster vs Vector graphics, Bit depth, Resolution concepts & 1 \\
    \hline
    1.4 & \textbf{Popular File Formats} \newline BMP, GIF, PNG, TIFF, JPEG - Structure and use cases & 1 \\
    \hline
    1.5 & \textbf{Color Science Fundamentals} \newline Human visual system, Color perception, Color spaces & 1 \\
    \hline
    1.6 & \textbf{Color Models in Images} \newline RGB, CMY/CMYK, HSV/HSL, YUV/YCbCr color models & 1 \\
    \hline
    1.7 & \textbf{Digital Audio Fundamentals} \newline Digitization of sound, Sampling, Quantization, Nyquist theorem & 1 \\
    \hline
    1.8 & \textbf{MIDI and Audio Interfaces} \newline Musical Instrument Digital Interface, MIDI messages, Synthesis & 1 \\
    \hline
    1.9 & \textbf{Digital Video Concepts} \newline Video signals, Frame rates, Interlacing, Video formats overview & 1 \\
    \hline
\end{longtable}

\vspace{0.3cm}
\noindent\textbf{Learning Outcomes:}
\begin{itemize}[leftmargin=*]
    \item Explain multimedia system components and their relationships
    \item Differentiate between various image file formats and their applications
    \item Apply appropriate color models for different multimedia applications
    \item Understand digital audio and video signal characteristics
\end{itemize}

\newpage

%----------------------------------------------------------------------
\subsection*{}
\modulebox{MODULE 2: Compression Algorithms \& Techniques}{9}
\addcontentsline{toc}{subsection}{Module 2: Compression Algorithms \& Techniques (9 Hours)}

\vspace{0.5cm}

\noindent\textbf{Overview:} This module covers both lossless and lossy compression algorithms, providing the theoretical foundation and practical implementation aspects of data compression.

\vspace{0.3cm}

\begin{longtable}{|c|p{10cm}|c|}
    \hline
    \rowcolor{lightblue}
    \textbf{Unit} & \textbf{Topic} & \textbf{Hours} \\
    \hline
    \endfirsthead
    \hline
    \rowcolor{lightblue}
    \textbf{Unit} & \textbf{Topic} & \textbf{Hours} \\
    \hline
    \endhead
    
    2.1 & \textbf{Information Theory Basics} \newline Entropy, Redundancy, Compression fundamentals & 1 \\
    \hline
    2.2 & \textbf{Run-Length Encoding (RLE)} \newline Algorithm, Applications, Variants for different data types & 1 \\
    \hline
    2.3 & \textbf{Variable-Length Coding} \newline Huffman coding, Shannon-Fano coding, Optimal prefix codes & 1 \\
    \hline
    2.4 & \textbf{Dictionary-Based Coding} \newline LZ77, LZ78, LZW algorithms, Sliding window techniques & 1 \\
    \hline
    2.5 & \textbf{Arithmetic Coding} \newline Principles, Encoder/Decoder implementation, Adaptive methods & 1 \\
    \hline
    2.6 & \textbf{Lossy Compression Fundamentals} \newline Distortion measures (MSE, PSNR), Rate-Distortion theory & 1 \\
    \hline
    2.7 & \textbf{Quantization Techniques} \newline Scalar quantization, Vector quantization, Lloyd-Max algorithm & 1 \\
    \hline
    2.8 & \textbf{Transform Coding} \newline DCT, DFT fundamentals, Transform domain compression & 1 \\
    \hline
    2.9 & \textbf{Wavelet-Based Coding} \newline Wavelet transforms, Multi-resolution analysis, Wavelet packets & 1 \\
    \hline
\end{longtable}

\vspace{0.3cm}
\noindent\textbf{Learning Outcomes:}
\begin{itemize}[leftmargin=*]
    \item Calculate entropy and compression bounds using information theory
    \item Implement lossless compression algorithms (Huffman, LZW, Arithmetic)
    \item Apply quantization techniques for lossy compression
    \item Understand transform-based and wavelet-based compression methods
\end{itemize}

\newpage

%----------------------------------------------------------------------
\subsection*{}
\modulebox{MODULE 3: Multimedia Compression Standards}{9}
\addcontentsline{toc}{subsection}{Module 3: Multimedia Compression Standards (9 Hours)}

\vspace{0.5cm}

\noindent\textbf{Overview:} This module focuses on industry-standard compression formats for images, audio, and video, with emphasis on JPEG and MPEG families.

\vspace{0.3cm}

\begin{longtable}{|c|p{10cm}|c|}
    \hline
    \rowcolor{lightblue}
    \textbf{Unit} & \textbf{Topic} & \textbf{Hours} \\
    \hline
    \endfirsthead
    \hline
    \rowcolor{lightblue}
    \textbf{Unit} & \textbf{Topic} & \textbf{Hours} \\
    \hline
    \endhead
    
    3.1 & \textbf{JPEG Standard} \newline Baseline JPEG, DCT-based compression, Quality settings & 1.5 \\
    \hline
    3.2 & \textbf{JPEG2000 \& JPEG-LS} \newline Wavelet-based JPEG2000, Lossless JPEG, Comparison & 1 \\
    \hline
    3.3 & \textbf{Bi-level Image Compression} \newline JBIG, JBIG2, Fax standards (G3, G4) & 0.5 \\
    \hline
    3.4 & \textbf{Audio Compression Fundamentals} \newline ADPCM, Speech coding principles, Vocoders & 1 \\
    \hline
    3.5 & \textbf{Psychoacoustic Models} \newline Human auditory system, Masking effects, Critical bands & 1 \\
    \hline
    3.6 & \textbf{MPEG Audio (MP3, AAC)} \newline MPEG-1 Layer III, AAC, Encoding/Decoding pipeline & 1 \\
    \hline
    3.7 & \textbf{Video Compression Principles} \newline Motion estimation, Motion compensation, Block matching & 1 \\
    \hline
    3.8 & \textbf{MPEG-1 \& MPEG-2} \newline Video bitstream structure, I/P/B frames, Interlaced support & 1 \\
    \hline
    3.9 & \textbf{MPEG-4, MPEG-7 \& Modern Codecs} \newline Object-based coding, Metadata, H.264/AVC overview & 1 \\
    \hline
\end{longtable}

\vspace{0.3cm}
\noindent\textbf{Learning Outcomes:}
\begin{itemize}[leftmargin=*]
    \item Explain the complete JPEG encoding/decoding pipeline
    \item Compare different image compression standards and their use cases
    \item Analyze MPEG audio and video compression techniques
    \item Evaluate trade-offs between compression ratio and quality
\end{itemize}

\newpage

%----------------------------------------------------------------------
\subsection*{}
\modulebox{MODULE 4: Advanced Applications \& Cloud Multimedia}{9}
\addcontentsline{toc}{subsection}{Module 4: Advanced Applications \& Cloud Multimedia (9 Hours)}

\vspace{0.5cm}

\noindent\textbf{Overview:} This module covers modern applications including content-based retrieval systems and cloud computing for multimedia services.

\vspace{0.3cm}

\begin{longtable}{|c|p{10cm}|c|}
    \hline
    \rowcolor{lightblue}
    \textbf{Unit} & \textbf{Topic} & \textbf{Hours} \\
    \hline
    \endfirsthead
    \hline
    \rowcolor{lightblue}
    \textbf{Unit} & \textbf{Topic} & \textbf{Hours} \\
    \hline
    \endhead
    
    4.1 & \textbf{Content-Based Image Retrieval (CBIR)} \newline Feature extraction, Color histograms, Texture features & 1.5 \\
    \hline
    4.2 & \textbf{Similarity Measures \& Indexing} \newline Distance metrics, Indexing structures, Query processing & 1 \\
    \hline
    4.3 & \textbf{CBIR Case Study} \newline CBIRD system, Practical implementation considerations & 1 \\
    \hline
    4.4 & \textbf{Video Retrieval \& Search} \newline Video segmentation, Keyframe extraction, Video querying & 1 \\
    \hline
    4.5 & \textbf{Cloud Computing Overview} \newline Cloud service models (IaaS, PaaS, SaaS), Deployment models & 1 \\
    \hline
    4.6 & \textbf{Multimedia Cloud Computing} \newline Architecture, Components, Resource management & 1 \\
    \hline
    4.7 & \textbf{Cloud Media Services} \newline Media sharing, Streaming services, CDN integration & 1 \\
    \hline
    4.8 & \textbf{Computation Offloading} \newline Mobile cloud computing, Service partitioning, Video coding & 1 \\
    \hline
    4.9 & \textbf{Interactive Cloud Gaming} \newline Architecture, Latency challenges, Quality of Experience & 0.5 \\
    \hline
\end{longtable}

\vspace{0.3cm}
\noindent\textbf{Learning Outcomes:}
\begin{itemize}[leftmargin=*]
    \item Design content-based image retrieval systems
    \item Implement similarity-based search for multimedia databases
    \item Analyze cloud computing architectures for multimedia applications
    \item Evaluate computation offloading strategies for mobile multimedia
\end{itemize}

\newpage

%======================================================================
\section{Teaching Plan Summary}
%======================================================================

\begin{center}
\begin{tabular}{|l|c|c|}
    \hline
    \rowcolor{headerblue}
    \textcolor{white}{\textbf{Module}} & \textcolor{white}{\textbf{Topics}} & \textcolor{white}{\textbf{Hours}} \\
    \hline
    Module 1: Multimedia Fundamentals & 9 & 9 \\
    \hline
    Module 2: Compression Algorithms & 9 & 9 \\
    \hline
    Module 3: Compression Standards & 9 & 9 \\
    \hline
    Module 4: Advanced Applications & 9 & 9 \\
    \hline
    \rowcolor{lightblue}
    \textbf{Total} & \textbf{36} & \textbf{36} \\
    \hline
\end{tabular}
\end{center}

%======================================================================
\section{Assessment Pattern}
%======================================================================

\subsection{Bloom's Taxonomy Distribution}
\begin{center}
\begin{tabular}{|l|c|c|c|}
    \hline
    \rowcolor{headerblue}
    \textcolor{white}{\textbf{Bloom's Category}} & \textcolor{white}{\textbf{Test 1 (\%)}} & \textcolor{white}{\textbf{Test 2 (\%)}} & \textcolor{white}{\textbf{ESE (\%)}} \\
    \hline
    Remember & 20 & 20 & 20 \\
    \hline
    Understand & 40 & 40 & 40 \\
    \hline
    Apply & 25 & 25 & 25 \\
    \hline
    Analyze/Evaluate & 15 & 15 & 15 \\
    \hline
\end{tabular}
\end{center}

\subsection{Mark Distribution}
\begin{center}
\begin{tabular}{|c|c|c|c|}
    \hline
    \rowcolor{lightblue}
    \textbf{Total Marks} & \textbf{CIE Marks} & \textbf{ESE Marks} & \textbf{ESE Duration} \\
    \hline
    150 & 50 & 100 & 3 Hours \\
    \hline
\end{tabular}
\end{center}

\subsection{Continuous Internal Evaluation (CIE)}
\begin{itemize}[leftmargin=*]
    \item \textbf{Attendance:} 10 marks
    \item \textbf{Continuous Assessment Tests:} 25 marks (Average of Series Tests 1 \& 2)
    \item \textbf{Assignments/Mini Projects:} 15 marks
\end{itemize}

\subsection{Internal Examination Pattern}
\begin{itemize}[leftmargin=*]
    \item \textbf{Part A:} 5 questions × 3 marks = 15 marks (Answer all)
    \item \textbf{Part B:} 7 questions × 7 marks, answer any 5 = 35 marks
    \item \textbf{Total:} 50 marks
\end{itemize}

\subsection{End Semester Examination Pattern}
\begin{itemize}[leftmargin=*]
    \item \textbf{Part A:} 8 questions (2 from each module) × 3 marks = 24 marks (Answer all)
    \item \textbf{Part B:} 8 questions (2 from each module), answer one from each module × 19 marks = 76 marks
    \item \textbf{Total:} 100 marks
\end{itemize}

\newpage

%======================================================================
\section{Text Books}
%======================================================================

\begin{enumerate}[leftmargin=*]
    \item \textbf{Ze-Nian Li and M. S. Drew}, \textit{Fundamentals of Multimedia}, 2nd Edition, Pearson Education, 2014. [Primary Text]
\end{enumerate}

%======================================================================
\section{Reference Books}
%======================================================================

\begin{enumerate}[leftmargin=*]
    \item K. R. Rao, Zoran S. Bojkovic, D. A. Milovanovic, \textit{Introduction to Multimedia Communications: Applications, Middleware, Networking}, Wiley, 2006.
    \item V. S. Subrahmanian, \textit{Principles of Multimedia Database Systems}, Morgan Kaufmann Publishers, 1998.
    \item R. Steinmetz and K. Nahrstedt, \textit{Multimedia: Computing, Communications and Applications}, Pearson Education, 2002.
    \item John F. Koegel Buford, \textit{Multimedia Systems}, Pearson Education, 1994.
    \item Prabhat K. Andheigh, Kiran Thakrar, \textit{Multimedia Systems Design}, Prentice Hall PTR, 1996.
    \item Jerry D. Gibson, \textit{Multimedia Communications: Directions and Innovations}, Elsevier Science, 2000.
\end{enumerate}

\newpage

%======================================================================
\section{Sample Question Bank}
%======================================================================

\subsection{Module 1 Questions}
\begin{enumerate}[leftmargin=*]
    \item Differentiate between multimedia and hypermedia with examples.
    \item Explain the differences between raster and vector graphics.
    \item Describe the RGB and YUV color models. When is each preferred?
    \item Explain the Nyquist theorem and its importance in audio digitization.
    \item Describe the structure of a MIDI message and explain Pitch Bend operation.
\end{enumerate}

\subsection{Module 2 Questions}
\begin{enumerate}[leftmargin=*]
    \item Calculate the entropy for a source with symbols having probabilities 0.4, 0.3, 0.2, 0.1.
    \item Construct a Huffman code for the symbols A, B, C, D with frequencies 5, 3, 2, 1.
    \item Explain the LZW algorithm with an example encoding of "ABABABAB".
    \item Describe the working of an Arithmetic encoder and decoder for input "01111".
    \item Compare scalar and vector quantization with advantages and disadvantages.
\end{enumerate}

\subsection{Module 3 Questions}
\begin{enumerate}[leftmargin=*]
    \item Describe the complete JPEG compression pipeline with a block diagram.
    \item Compare JPEG and JPEG2000 standards in terms of features and performance.
    \item Explain psychoacoustic masking and its application in MPEG audio compression.
    \item Calculate the compression ratio for MPEG audio at 256 kbps (16-bit, 48 kHz stereo).
    \item Draw and explain the MPEG-2 encoder and decoder block diagrams.
\end{enumerate}

\subsection{Module 4 Questions}
\begin{enumerate}[leftmargin=*]
    \item Explain content-based image retrieval with feature extraction techniques.
    \item How do you evaluate the performance of an image search engine?
    \item Differentiate between public, private, and hybrid cloud deployments.
    \item Explain the architecture of multimedia cloud computing with all components.
    \item Discuss the challenges and requirements for computation offloading in mobile multimedia.
\end{enumerate}

\newpage

%======================================================================
\section{Model Question Paper}
%======================================================================

\begin{center}
    \textbf{APJ ABDUL KALAM TECHNOLOGICAL UNIVERSITY}\\
    \textbf{B.TECH DEGREE EXAMINATION}\\[0.3cm]
    \textbf{Course Code: CXT332}\\
    \textbf{Course Name: Multimedia Technologies}\\[0.3cm]
    \textbf{Max Marks: 100 \hspace{2cm} Duration: 3 Hours}
\end{center}

\vspace{0.5cm}
\hrule
\vspace{0.5cm}

\noindent\textbf{PART A}\\
\textit{(Answer all questions. Each question carries 3 marks)}\\[0.3cm]

\begin{enumerate}
    \item Differentiate between multimedia and hypermedia.
    \item Explain the RGB color model with its applications.
    \item What is the significance of entropy in data compression?
    \item Compare Huffman coding and Arithmetic coding.
    \item Describe the DCT transform used in JPEG.
    \item What is psychoacoustic masking?
    \item Explain I, P, and B frames in MPEG video.
    \item What is content-based image retrieval?
    \item Define cloud service models (IaaS, PaaS, SaaS).
    \item What is computation offloading?
\end{enumerate}
\hfill \textbf{(8 × 3 = 24 Marks)}

\vspace{0.5cm}
\noindent\textbf{PART B}\\
\textit{(Answer one question from each module. Each question carries 19 marks)}\\[0.3cm]

\noindent\textbf{Module 1}
\begin{enumerate}[start=11]
    \item (a) Explain various color models used in image processing. (10)\\
          (b) Describe the digitization process of audio signals. (9)
    \item (a) Discuss different multimedia file formats and their characteristics. (10)\\
          (b) Explain the MIDI protocol and its applications. (9)
\end{enumerate}

\noindent\textbf{Module 2}
\begin{enumerate}[start=13]
    \item (a) Explain Huffman coding with an example. (10)\\
          (b) Describe wavelet-based compression techniques. (9)
    \item (a) Explain the LZW compression algorithm with an example. (10)\\
          (b) Describe vector quantization and its advantages. (9)
\end{enumerate}

\noindent\textbf{Module 3}
\begin{enumerate}[start=15]
    \item (a) Describe the JPEG compression standard in detail. (12)\\
          (b) Compare JPEG and JPEG2000 standards. (7)
    \item (a) Explain MPEG-1 video compression with block diagram. (12)\\
          (b) Describe MPEG audio compression. (7)
\end{enumerate}

\noindent\textbf{Module 4}
\begin{enumerate}[start=17]
    \item (a) Explain content-based image retrieval techniques. (10)\\
          (b) Describe similarity-based search in multimedia databases. (9)
    \item (a) Explain multimedia cloud computing architecture. (10)\\
          (b) Discuss computation offloading for multimedia services. (9)
\end{enumerate}
\hfill \textbf{(4 × 19 = 76 Marks)}

\vspace{1cm}
\begin{center}
    \rule{0.5\textwidth}{0.5pt}\\
    \textit{End of Question Paper}
\end{center}

\end{document}
